\par The Acadia National Park Final Transportation Plan is being implemented with the purpose of improving traffic and parking conditions while ensuring the quality of visitor experience and natural resources are well protected.  This project sought to verify the feasibility of establishing a comprehensive monitoring system to fulfill the Transportation Plan while including the recommendations we gave to the Park Rangers.  A key background review that describes park conditions, traffic congestion, the Final Transportation Plan, and related projects was performed in the following sections.

\section{Description of the Park}
\par Acadia National Park is located across Mount Desert Island, the Schoodic Peninsula, Isle Au Haut, and many smaller coastal islands around Bar Harbor, Maine, some only a few meters across. About 30,000 acres is located on Mount Desert Island, constituting 60 percent of the total area. Originally established in 1916 as the Sieur de Monts National Monument by Woodrow Wilson, its purpose was to protect the natural beauty along the Atlantic shore of the United States. In 1929, the park was renamed to Acadia as a condition of a large private land donation on the Schoodic Peninsula.  Figure 1 below shows the location and scale of Acadia National Park relative to Mount Desert Island and the State of Maine.

\nfigure{location_of_park}{Location of Acadia National Park}

\par Even though the park has received many land donations throughout history, today it still remains one of the smallest National Parks with an area of fewer than 50 thousand acres. Despite this, it is one of the ten most popular national parks, with the number of visitors increased by 58.8 percent since 2009 to a staggering 3.54 million with most attending between the months of June and September (NPS 2018). Visitors can enjoy recreational activities including viewing the scenery, hiking, camping, bicycling, and canoeing. Within the park, several locations including Cadillac Mountain, Sand Beach, and Jordan Pond make up the most popular attractions, drawing the bulk of visitors, and directly contributing to the congestion issues which currently plague the park (NPS, Traffic and Congestion, 2019).

\section{Traffic Congestion and Safety Concerns}
\par As Acadia National Park has become more popular, the existing road infrastructure has begun to struggle with the number of visitors attending each year. Many of the roads, constructed as many as one hundred or more years ago have become overwhelmed with recreational vehicles, pedestrians, cyclists, the Island Explorer and tour busses. This poses a serious threat to visitor safety, as emergency vehicles can become encumbered while attempting to navigate through masses of illegally parked vehicles in route to the scene of an emergency site. The massive overcrowding of vehicles also detracts greatly from visitor experience, forcing visitors to wait for traffic, skip attractions, or park illegally.

\par As annual visitation rates have continued to rise over the past 28 years, the parks limited infrastructure has been forced to withstand 51 percent more traffic, contributing to an exponential increase in congestion (National Park Service, 2018).  In 2005, the park experienced just over two million visitors, however by 2018 that number had increased to nearly 3.6 million.

\par Previous research into the subject of traffic congestion in Acadia is plentiful, having been the subject of numerous academic analyses over the past 20 years.  In “Preparing Acadia National Park for Modern Tourist Congestion” (Cosmopulos, Gaulin, Jauris, Morisseau, Quevillon), determined that traffic congestion posed a serious issue during several peak time slots, during which the number of visitors could increase by as much as 600 percent in less than an hour.

\par One major contributing factor to the congestion is the intrinsic link between automobiles and the National Park Service. For many visitors to the park, "Ocean Drive is important because it allows access to and enjoyment of the scenery" (Hallo, Manning, 494).  In total, 96 percent of surveyed respondents claimed that driving was an important part of the park experience (Hallo, Manning, 494). With visitor traffic increasing at an ever greater rate, and the great American pastime of driving not predicted to go anywhere anytime soon, congestion is only projected to get worse unless something is done.

\par Cadillac Mountain is the most crowded point for visitors to Acadia National Park since visitors can enjoy magnificent views of dotted island landscape from all directions. Cadillac Summit is only accessible through a twisting and narrow road which is closed from December to April. Studies indicated that 75 percent of visitors went to Cadillac Mountain Area during their stay (Manning 2010). Between June 2,2016 and October 31,2016, about 259,000 vehicles and 777,000 visitors went to Cadillac Summit (NPS 2017b). The road to the summit was closed 70 times in 2017 due to congestion issues (Dolores Kong, Dan Ring 2018). 

\section{Final Transportation Plan}
\par Since the congestion problems created safety issues, natural and historic resources protection concerns, and adverse impacts on visitor experience, the National Park Service(NPS) released the Final Transportation Plan and accompanying Environmental Impact Statement (Final Plan/EIS) on March, 2019 for Acadia National Park (NPS, 2019). Final Plan aims at giving a comprehensive approach to provide safe and efficient transportation to visitors and protect park resources and values (NPS, 2019). The following are the key actions related to project. 

\subsubsection*{Reservation Systems}
\par It is one of the most critical actions that NPS made to react transportation issues. The purpose of the reservation system is to control football within the desired visitor capacities not including cyclists or pedestrians in hot scenic spots (Cadillac Summit, Thunder Hole, Jordan Pond, Sand Beach, Bubble Rock, etc.). Reservations can be made online or automatic service machine at Hulls Cove Visitor Center and Acadia Gateway Center. The number of reservations or the time length of parking reservation for each location would be adjusted in different time period to provide the best visitor experience while preventing traffic congestion and parking queue. If the volume of transit at specific location still exceeds set thresholds after the reservation systems for recreational vehicles are implemented, the reservation system could be extended to Island Explorer in order to manage traffic issues. 

\subsubsection*{Public Transit}
\par The Island Explorer service was a public transportation system with ten routes linking hotels, popular attractions (except Cadillac Summit), campgrounds, and visitor centers on Mount Desert Island and the Schoodic Peninsula (Island Explorer Home, 2019). Since the Island Explorer service cannot meet demands at peak volume of transit, increasing the number of Island Explorer buses becomes another important component of the Transportation Plan. As described in the last section, Island Explorer service would also be expanded to cohere with the reservation systems requirements. 

\subsubsection*{Visitor Information}
\par The National Park Service planned to provide more varied and detailed information to visitors, both before and after they arrive at the park. Those increased information includes reservation requirements and availability, trip-planning tools, orientation tips, advice on vehicle/ bicycle safety, and car-free road sections. Information about traffic congestion and parking availability would be monitored and disseminated to the public, and it can be accessed online or through mobile applications. Additionally, the park would improve cellular connectivity throughout the park by cooperating with communication providers. With better cellular signal, visitors have the capability to receive more real-time information from the park as a reference for planning their trips.

\subsection{Planned Monitoring Strategies}
\par The National Park Service would adopt the Visitor Use Management Framework created by the Interagency Visitor Use Management Council (IVUMC, 2016). This Framework includes establishing indicators, thresholds, and determining visitor capacity, and monitoring is used to track these conditions over time (NPS, 2019). According to monitoring results, park rangers would adjust the management strategies to ensure that the amount of visitor use will not exceed the maximum threshold. The following list below is several Monitoring Strategies that would be implemented in the plan.

\subsubsection*{Vehicles at one time (VAOT)}
\par It is a measurement commonly used by park rangers to quantify the amount of vehicles in parking lots and right-lane parking (Manning, Lawson et al.2014). VAOT will be used as an indicator of transit and access conditions at crowded attractions (e.g., Thunder Hole, Jordan Pond, Sand Beach, Bubble Rock, Echo Lake Beach, Cadillac Summit). This indicator would also ensure the visitors with reservations of parking lots can arrive at their intended destinations as assigned.

\subsubsection*{People per Viewscape (PPV)}
\par PPV is a measure that helps park managers to quantify visitor crowding impacts along hiking trails, walking paths, and other scenic corridors in national parks (Lawson et al. 2011; Lawson et al. 2009; Manning et al. 2011; NPS, 2019). PPV will be monitored on at least one of the following higher use trails: Jordan Pond Path, Beehive Trail, South Bubble Trail, Wonderland Trail, Cadillac Mountain Gorge path, Schoodic Head Trail (NPS, 2019).

\subsubsection*{Automated vehicle traffic recorder (ATR)}
\par It is a device for collecting large amounts of traffic volume data. ATR can measure the direction of flow, traffic speed, and other crucial variables through pneumatic tubes (Accu-Traffic Inc., 2010). Based on ATR data, NPS planned to establish statistical and mathematical relationships with other indicators like VAOT, vehicles percent time following, and PPV for visitor use management.

\subsection{Webcams as an Additional Monitoring Strategy}
\par The monitoring strategies which would be implemented in the plan are all measurable attributes (e.g., VAOT, PPV) that can help park rangers to make changes in resources and conditions related to visitor experience, but these data cannot describe specific situation of any road conditions or parking lots. In that case, webcams at key locations are more intuitive approach to tell what happened compared to numerical information. For emergency case, the real-time images sending back from webcams can help shorten the reaction time and make the most appropriate decisions. Webcams can also be conducted as an indicator by importing the image recognition and analyzing algorithm so that computers can automatically output the conditions of roads or parking lots.

\subsection{Webcams as an Information Provider}
\par As described in the ‘Visitor information’ section, the National Park Service planned to provide information about congestion and parking availability and enhanced trip-planning tools to visitors. Webcams are great information source by uploading the real-time images to the public website so that these images can be accessible to the public. With better cellular connection in the future, visitors could browse the conditions of road and parking lots of all the attractions upon arrival so that visitors can adjust their schedule to avoid traffic congestion. Visitors would have a more comprehensive picture of the transit conditions of the park by browsing images of previous years before they arrive at the park.

\section{Related Project Overview}
\par This project is an extension of 2018 Acadia Webcam IQP. In 2018, previous team members explored the “Feasibility of Webcam Implementation in Acadia National Park”. They benchmarked the webcams in Yellowstone National Park and decided to choose webcams as a solution for the traffic congestion in Acadia National Park. In order to prove this idea is executable, they also did research on the cost of webcams as well as the visitor rights and ethics to prove. It turned out that webcams were not expensive and monitoring in the park was unlikely to be a violation of privacy rights. Thus, they used the Spypoint Link-S camera to take pictures in fourteen locations in the park and successfully got decent pictures. After determining the camera’s battery performance, cell signal and picture quality, they concluded “a webcam could be a reliable tool to remotely monitor traffic in parking lots” (Bruno, Cromwick, \& Feng, 2018). They also suggested the connectivity and power source might be a limitation that need to be worked on.

\par Besides the 2018 Acadia Webcam Team, this project also took the works from 2018 Acadia Cellular Connectivity Team and 2018 Glacier Webcam Team as references. The 2018 Acadia Cellular Connectivity Team measured cellular connectivity status in Acadia National Park. They recorded the cellular signal strength and internet speed of different providers throughout the island. The final cellular signal heat maps suggested AT\&T is the most reliable and consistent signal provider on Mt Desert Island (Bergquist, Palacios, \& Goklevent, 2018). On the other hand, 2018 Glacier Webcam Team explored the feasibility using Raspberry Pi as the platform of webcams. They provided a complete tutorial on how to configure FTP (File Transfer Protocol) parameters (Barrameda, Vose, \& Rizzo, 2018). This would allow the pictures taken by the webcams upload to a website directly.

\nfigure{cell_signal_heat_map}{AT\&T Cellular Signal Heat Map}