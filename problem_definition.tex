\nchapter{Problem Definition}

\par The sudden and rapid spread of COVID-19 has had many unforeseen impacts on our society.  Many people are now working from their homes, avoiding public gatherings, and minimizing their number of weekly shopping trips.  As a result of this, bulk purchases of various essential items have become increasingly common, causing shortages of said items, which in turn scare people into rushing out to purchase resulting in a vicious and self-sustaining cycle.\\

\par Of all common household items, none has been stuck by this fate harder than the humble roll of toilet paper.  Fear has swept the world as this once common item has become scarce.  In Australia, two women were charged with affray after assaulting another patron over limited supplies \cite{probdef_australia}.  Elsewhere in the world, opportunists are attempting to score a quick buck by acquiring large quantities of bathroom tissue and reselling it online \cite{probdef_california}.\\

\par Given the scarcity of Toilet Paper as a critical resource, now more than ever it is desirable to have control over the quantity of toilet paper being consumed by members of a household.  The average American uses approximately 23.6 rolls of toilet paper per year \cite{kaufman_2009}, or approximately one roll every other week.  With an average of 2.63 people per household \cite{probdef_census}, that equates to approximately 62.1 rolls of paper per year, or 1.2 rolls per week.\\

\par This problem of controlling access to bathroom tissue has been recognized since long before the advent of the so-called "Internet of Things".  Devices for controlling access to toilet paper and other rolled paper products have existed since as early as 1965 \cite{mcgrew_tpdisp} with varying degrees of complexity, with some featuring suites of sensors and electronic controls \cite{kclark_disp}.  Despite existing technology, no device manufacturer has yet connected this technology to the internet, or created a system by which tissue usage is strictly rationed.