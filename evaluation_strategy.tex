\nchapter{Evaluation Strategy}
To properly evaluate this project's ability to provide toilet paper as a service, we as a team have come up with a few metrics that we would like to use to see how TPaaS performs both qualitatively and quantitatively. These Metrics are outlines as follows.
\nsection{Reliability}
Reliability is an important aspect of this project. Our system needs to be stable enough that it can properly dispense the right amount of toilet paper when demanded without a problem. Beyond that, the reliability and stability of our app and embedded computer architecture need to be solid enough to be able to handle any edge cases that may arise, such as a sensor misread or servo abuse (from a user forcing a servo to move therefore dispensing toilet paper without the app).

To evaluate the reliability of our project, we will be performing a variety of tests within the code to see if the system performs as expected. As our codebase is being written in Python\cite{python}, we will be utilizing Tox\cite{tox} to test the reliability of our code. Tox is a tool that works in conjunction with PyLint\cite{pylint} and Flake8\cite{flake8} to perform unit testing on the code to make sure that it is standardized between developers and will run as expected. 

\nsection{Accuracy}
Accuracy is also an important aspect of this project. As the main purpose of this system is to tell the user how many rolls they have in their TP Vault and to dispense a roll when requested, it is important that the system is accurate and tells the user both the correct number of rolls and only dispenses one roll when requested.

as such, we will also be performing real-world edge case testing on the system to ensure that it performs as expected and will work when the unexpected happens. Some examples of ways that we want to test this are by loading partially used Toilet Paper rolls into the system. doing so will ensure that the code based fail-safes are properly inputted into the system. 
\nsection{Scalability}
Additionally, Scalability is also an important goal of this project. as such, we want to be able to test how scalable our system is. As such, we will be developing methods to make the system easier to implement for users so that it can be added to other households or better yet commercial spaces such as WPI. 
TO perform this testing, we will be looking into the memory and CPU usage that our programs use on both the mobile device and on the embedded computer. this will allow us to see if our system can work on lower resource systems such as a Particle Photon or a Raspberry Pi 0.
\nsection{Ease of Use}
Finally, Ease of use is critical to this system. as TPaaS is aimed primarily at residential users, it is important that people of all ages can operate this system. as such, we will be conducting tests upon family members with the household where the prototype exists. this will act as a basic surface level test to see how others respond to using our system. As we cannot responsibly interact with other people outside of our household, we have refrained from including them in our testing infrastructure. If this pandemic does clear up before the end of the term, we will reevaluate accordingly and report back.