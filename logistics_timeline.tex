\nchapter{Project Logistics, Timeline, and Milestones}

\par The current situation regarding COVID-19 and WPI's decision to move all work off campus, the logistics of developing a project in a group setting have been greatly complicated.  To address this, we have implemented an organizational structure which allows us to effectively work as a team despite the physical separation.  Primary development, including the construction of the primary prototype, will be undertaken at the residence of Bryce S. Corbitt.  The software development process will be a collective process undertaken by all members of the team facilitated by git\footnote{https://git-scm.com/}, and hosted on GitHub\footnote{https://github.com/ECE2305-Group-25}.  Additional prototyping and experimentation for components will be undertaken by Nicholas H. Hollander.

\par In order to ensure that a fully functional prototype device is ready by the conclusion of the term, we have pre-established a timeline of events outlining our desired progress and areas of focus.  This process is outlined in Figure \ref{fig:timeline}, and is broken down as follows.

\nfigure{timeline}{Proposed Design and Development Timeline}

\subsubsection*{Week 1: Brainstorming}
\par Before beginning on our journey to implement the worlds next revolutionary IoT device, we first had to come up with an idea.  A meeting was held in which we evaluated the current state of the world, and determined areas of focus most suitable for the creation of an internet connected solution.  This process yielded several areas of focus related to the current COVID-19 epidemic, including potential systems for enhancing social distancing, controlling and mitigating disease transmission vectors, and management of low-availability critical resources.

\subsubsection*{Week 2: Continued Brainstorming}
\par Following the selection of our areas of focus, we explored a variety of applied solutions to common issues.  As toilet paper shortages were fresh in the news due to wide scale hoardings, we chose to build our solution around the management of this critical resource.  Further brainstorming produced the idea of storing rolls of toilet paper in a secure queue to restrict access to only authenticated people via an application installed on or accessed via a users cell phone.

\subsubsection*{Week 3: Start Prototyping}
\par With our idea fully fleshed out, we are now ready to begin the prototyping phase.  Material and component selections are being performed, and an implementable design is being finalized.  By the conclusion of this week, the parts and components of our device will be selected, and ready for assembly and wiring.

\subsubsection*{Week 4: Continue Prototyping}
\par This week will focus on the assembly and completion of the prototype.  Due to the limited access to engineering resources, it is probable that a variety of changes will need to be made to the components of the device, which will all be completed during this period.

\subsubsection*{Week 5: Testing and Debugging}
\par At this point the marriage of software and hardware will take place.  Control code, server code, and application code will all be connected to the system, and we will have dispensed our first toilet paper roll successfully.

\subsubsection*{Week 6: Present our Final Product}
\par At this point Toilet paper As A Service will have been realized, and we will have nothing other than the finest in totalitarian toilet tissue management ready for your judgment.

\par The other major measurement of our progress and success will be characterized by our achievement of several notable milestones, defined below.

\subsubsection*{Milestone 1: First roll dispensed (no IoT)}
\par The first major milestone in our project will be the first roll of toilet paper which is successfully released from the secure containment vessel via the built in electromechanical actuation systems.  Although at this point there is no server connection, this is still a major achievement.

\subsubsection*{Milestone 2: Marriage of Technology}
\par The next major milestone will be the marriage of technology, bringing together server code, embedded code, and cellular application code to allow a user's request for a roll of paper to seamlessly travel through the chain of devices down to the dispenser itself.  Successful transmission of messages between these points will be our next major achievement.

\subsubsection*{Milestone 3: Remote release of TP}
\par The next and final major milestone will be the first roll of paper released from the secure containment vessel following an authenticated user request from a mobile device.  Representing the completion of our project, this is the most significant of all three milestones.
